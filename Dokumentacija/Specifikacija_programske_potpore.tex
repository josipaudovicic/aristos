\chapter{Specifikacija programske potpore}
		
	\section{Funkcionalni zahtjevi}
			
			\noindent \textbf{Dionici:}
			
			\begin{packed_enum}
				\item Tragač na terenu
				\item Istraživač				
				\item Voditelj postaje
				\item Administrator
				
			\end{packed_enum}
			
			\noindent \textbf{Aktori i njihovi funkcionalni zahtjevi:}
			
			
			\begin{packed_enum}
				\item  \underbar{Neregistrirani korisnik  (inicijator) može:}
				
				\begin{packed_enum}
					\item se registrirati u sustav
					 \begin{packed_enum}
						\item  dati svoje podatke: korisničko ime, fotografija, lozinka, ime, prezime i email adresa
						\item  odabrati svoju ulogu(tragač, istraživač ili voditelj postaje)
						\item potvrditi registraciju na svojoj email adresi 
					\end{packed_enum}
				\end{packed_enum}
			
				\item  \underbar{Korisnik (inicijator) može:}
				
				\begin{packed_enum}
					\item se prijaviti u sustav
					\item upravljati svojim podatcima
					 \begin{packed_enum}
						\item  pregled osobnih podataka
						\item promjena osobnih podataka
						\item brisanje korisničkog računa
					\end{packed_enum}
				\end{packed_enum}

				\item  \underbar{Tragač na terenu (inicijator) može:}
				
				\begin{packed_enum}
					
					\item vidjeti zadatke tijekom akcije
					 \begin{packed_enum}
						\item označiti da je zadatak riješen
					\end{packed_enum}
					\item završiti akciju
					\item vidjeti gdje se nalaze ostali tragači na akciji 
					\item vidjeti koje su dostupne akcije 
					 \begin{packed_enum}
						\item prihvatiti akciju
						\item odbiti akciju
					\end{packed_enum} 
					\item vidjeti pozicije praćenih životinja
					\item odabrati vozilo kojim će ići u istraživanje
					\item vidjeti informacije o životinjama
					 \begin{packed_enum}
						\item dodati komentar
						\item izbrisati komentar
					\end{packed_enum}

					
				\end{packed_enum}

				\item  \underbar{Istraživač (inicijator) može:}
				
				\begin{packed_enum}

					\item vidjeti informacije o životinjama
					 \begin{packed_enum}
						\item dodati komentar
						\item izbrisati komentar
					\end{packed_enum}
					\item vidjeti kartu 
					 \begin{packed_enum}
						\item izraditi kartu: na temelju životinja ili tragača
					\end{packed_enum}
					\item stvoriti novu akciju
					 \begin{packed_enum}
						\item dodati nove zadatke
					\end{packed_enum}
					\item poslati voditelju postaje zahtjev za tragačima
					 \begin{packed_enum}
						\item dodati opis o željenim tragačima
					\end{packed_enum}
					\item zadati zadatke tragačima
					 \begin{packed_enum}
						\item upravljati zadacima
						 \begin{packed_enum}
							\item dodavati zadatke
							\item uređivati zadatke
							\item brisati zadatke
						\end{packed_enum}
						\item dodati komentare na zadatke
					\end{packed_enum}
					\item pregledati informacije


				\end{packed_enum}

				\item  \underbar{Voditelj postaje (inicijator) može:}
				
				\begin{packed_enum}
					\item pregledati sve tragače
					\item dodati tragača na akciju
					\item dodijeliti postaju tragaču
				\end{packed_enum}

				\item  \underbar{Administrator (inicijator) može:}
				
				\begin{packed_enum}
					\item vidjeti popis svih registriranih korisnika i njihovih osobnih podataka
					\item obrisati korisnika
					\item mijenjati dodijeljena prava i osobne podatke registriranim korisnicima
					\item potvrditi istraživača i voditelja postaje
				\end{packed_enum}

				\item  \underbar{Baza podataka (sudionik):}
				
				\begin{packed_enum}
					\item pohranjuje sve podatke o korisnicima 
					\item pohranjuje sve podatke o životinjama
					\item pohranjuje staze kojima tragači putuju(i način kojim su se kretali)
				\end{packed_enum}


			\end{packed_enum}
			
			\eject 
			
			
				
			\subsection{Obrasci uporabe}
								
				\subsubsection{Opis obrazaca uporabe}
					\textit{Funkcionalne zahtjeve razraditi u obliku obrazaca uporabe. Svaki obrazac je potrebno razraditi prema donjem predlošku. Ukoliko u nekom koraku može doći do odstupanja, potrebno je to odstupanje opisati i po mogućnosti ponuditi rješenje kojim bi se tijek obrasca vratio na osnovni tijek.}\\
					

					\noindent \underbar{\textbf{UC3 - Prijava u sustav}}
					\begin{packed_item}
	
						\item \textbf{Glavni sudionik: }Korisnik (voditelj postaje, tragač, istraživač)
						\item  \textbf{Cilj:} Pristup registriranog korisnika korisničkom sučelju
						\item  \textbf{Sudionici:} Baza podataka
						\item  \textbf{Preduvjet:} Registracija
						\item  \textbf{Opis osnovnog tijeka:} 
						
						
						\item[] \begin{packed_enum}
	
							\item Unos korisničkog imena i lozinke							
							\item Potvrda o ispravnosti unesenih podataka
							\item Pristup korisničkim funkcijama
							
						\end{packed_enum}
						
						\item  \textbf{Opis mogućih odstupanja:}
						
						\item[] \begin{packed_item}
	
							\item[2.a] Neispravno korisničko ime/lozinka
							\item[] \begin{packed_enum}
								
								\item Sustav obavještava korisnika o neuspjelom pokušaju i vraća ga na stranicu za registraciju 
								
							\end{packed_enum}
							%\item[2.b] $<$opis mogućeg scenarija odstupanja u koraku 2$>$
							%\item[3.a] $<$opis mogućeg scenarija odstupanja  u koraku 3$>$
							
						\end{packed_item}
					\end{packed_item}

					\noindent \underbar{\textbf{UC$<$broj obrasca$>$ -$<$ime obrasca$>$}}
					\begin{packed_item}
	
						\item \textbf{Glavni sudionik: }$<$sudionik$>$
						\item  \textbf{Cilj:} $<$cilj$>$
						\item  \textbf{Sudionici:} $<$sudionici$>$
						\item  \textbf{Preduvjet:} $<$preduvjet$>$
						\item  \textbf{Opis osnovnog tijeka:}
						
						\item[] \begin{packed_enum}
	
							\item $<$opis korak jedan$>$
							\item $<$opis korak dva$>$
							\item $<$opis korak tri$>$
							\item $<$opis korak četiri$>$
							\item $<$opis korak pet$>$
						\end{packed_enum}
						
						\item  \textbf{Opis mogućih odstupanja:}
						
						\item[] \begin{packed_item}
	
							\item[2.a] $<$opis mogućeg scenarija odstupanja u koraku 2$>$
							\item[] \begin{packed_enum}
								
								\item $<$opis rješenja mogućeg scenarija korak 1$>$
								\item $<$opis rješenja mogućeg scenarija korak 2$>$
								
							\end{packed_enum}
							\item[2.b] $<$opis mogućeg scenarija odstupanja u koraku 2$>$
							\item[3.a] $<$opis mogućeg scenarija odstupanja  u koraku 3$>$
							
						\end{packed_item}
					\end{packed_item}
				
					
				\subsubsection{Dijagrami obrazaca uporabe}
					
					\textit{Prikazati odnos aktora i obrazaca uporabe odgovarajućim UML dijagramom. Nije nužno nacrtati sve na jednom dijagramu. Modelirati po razinama apstrakcije i skupovima srodnih funkcionalnosti.}
				\eject		
				
			\subsection{Sekvencijski dijagrami}
				
				\textbf{\textit{dio 1. revizije}}\\
				
				\textit{Nacrtati sekvencijske dijagrame koji modeliraju najvažnije dijelove sustava (max. 4 dijagrama). Ukoliko postoji nedoumica oko odabira, razjasniti s asistentom. Uz svaki dijagram napisati detaljni opis dijagrama.}
				\eject
	
		\section{Ostali zahtjevi}
		
			\textbf{\textit{dio 1. revizije}}\\
		 
			 \textit{Nefunkcionalni zahtjevi i zahtjevi domene primjene dopunjuju funkcionalne zahtjeve. Oni opisuju \textbf{kako se sustav treba ponašati} i koja \textbf{ograničenja} treba poštivati (performanse, korisničko iskustvo, pouzdanost, standardi kvalitete, sigurnost...). Primjeri takvih zahtjeva u Vašem projektu mogu biti: podržani jezici korisničkog sučelja, vrijeme odziva, najveći mogući podržani broj korisnika, podržane web/mobilne platforme, razina zaštite (protokoli komunikacije, kriptiranje...)... Svaki takav zahtjev potrebno je navesti u jednoj ili dvije rečenice.}
			 
			 
			 
	