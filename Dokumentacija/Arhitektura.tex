\chapter{Arhitektura i dizajn sustava}

Arhitektura sustava može se podijeliti na tri ključna podsustava:

\begin{packed_enum}
	\item Web poslužitelj:
	\begin{packed_enum}
		\item Srce web aplikacije.
		\item Odgovoran za interakciju između klijenta i aplikacije.
		\item Koristi HTTP/HTTPS protokol za prijenos informacija na webu.
		\item Inicira pokretanje web aplikacije i proslijeđuje zahtjeve.
	\end{packed_enum}
	\item Web aplikacija:
	\begin{packed_enum}
		\item Procesira korisničke zahtjeve i obrađuje ih.
		\item Pristupa bazi podataka prema potrebi.
		\item Generira odgovore u obliku HTML dokumenata za prikaz u web pregledniku.
	\end{packed_enum}			
	\item Baza podataka:	
	\begin{packed_enum}
		\item Sprema podatke koji se koriste ili modificiraju unutar web aplikacije.
	\end{packed_enum}
\end{packed_enum}

Korisnik, putem web preglednika, šalje zahtjeve web poslužitelju. 
Web poslužitelj zatim inicira rad web aplikacije, koja procesira zahtjeve, pristupa bazi podataka po potrebi i vraća odgovore u obliku HTML dokumenata. 
Ova interakcija omogućuje korisnicima pregled i manipulaciju sadržajem putem web sučelja.

\begin{figure}[H]
	\includegraphics[scale=0.5]{slike/arhitektura.png}
	\centering
	\caption{Arhitektura sustava}
	\label{fig:arhitektura}
\end{figure}

React je biblioteka koju smo odabrali za izradu naše web aplikacije, zajedno sa Spring Boot radnim okvirom te programskim jezikom JavaScript. 
Odabrano razvojno okruženje je Visual Studio Code. Arhitektura sustava temeljiti će se na MVC(Model-View-Controller) konceptu.
\newline React je biblioteka otvorenog koda koja se koristi za izgradnju korisničkih sučelja.
\begin{packed_item}
	\item Jezik: Povezan je s JavaScriptom
	\item Slučajevi upotrebe: Često se koristi za izradu jednostranih aplikacija, gdje je potrebna brza i prilagodljiva interakcija
	\item Platforma: Nema specifičnu platformu; može se koristiti u web aplikacijama na različitim platformama
\end{packed_item}
Spring Boot je radno okruženje koje se koristi za izgradnju aplikacija.
\begin{packed_item}
	\item Jezik: prvenstveno povezan s Javom
	\item Radno okruženje: Spring Boot dio je Spring Frameworka za Java razvoj
	\item Slučajevi upotrebe: naširoko se koristi za izradu web aplikacija temeljenih na Javi
	\item Ekosustav: Java ekosustav, s jakom integracijom s tehnologijama kao što su Spring MVC, Spring Data itd.
	\item Platforma: Java Virtual Machine (JVM)
\end{packed_item}
Spring Boot podržava koncept MVC (Model-View-Controller) arhitekture, a to se postiže kroz Spring Web MVC modul. Spring Boot podržava MVC koncept:
\begin{packed_item}
	\item Model: Spring Boot omogućava korištenje Java objekata kao modela. Ovi objekti predstavljaju podatke koji se koriste u aplikaciji.
	Spring Data može se integrirati za jednostavno upravljanje podacima i komunikaciju s bazom podataka.
	\item View: Spring Boot pruža fleksibilnost u odabiru tehnologije za prikazivanje korisničkog sučelja. Prikazi se često implementiraju kroz HTML datoteke, a moguće je koristiti različite template engines (Thymeleaf, FreeMarker, JSP).
	Pomoću konfiguracija view resolvera jednostavno se integriraju odabrane tehnologije za prikazivanje podataka korisnicima.
	\item Controller: Anotacije poput @Controller i @RestController omogućuju jednostavno označavanje klasa koje djeluju kao kontroleri.
	@RequestMapping i slične anotacije omogućuju mapiranje HTTP zahtjeva na određene metode kontrolera.
	Spring Boot automatski prepoznaje i konfigurira komponente kontrolera.		
\end{packed_item}
Primjena MVC koncepta u Spring Boot-u olakšava održavanje i proširivost aplikacija.
	
				
		\section{Baza podataka}
			
			\textbf{\textit{dio 1. revizije}}\\
			
		\textit{Potrebno je opisati koju vrstu i implementaciju baze podataka ste odabrali, glavne komponente od kojih se sastoji i slično.}
		
			\subsection{Opis tablica}
			

				\textit{Svaku tablicu je potrebno opisati po zadanom predlošku. Lijevo se nalazi točno ime varijable u bazi podataka, u sredini se nalazi tip podataka, a desno se nalazi opis varijable. Svjetlozelenom bojom označite primarni ključ. Svjetlo plavom označite strani ključ}
				
				
				\begin{longtblr}[
					label=none,
					entry=none
					]{
						width = \textwidth,
						colspec={|X[6,l]|X[6, l]|X[20, l]|}, 
						rowhead = 1,
					} %definicija širine tablice, širine stupaca, poravnanje i broja redaka naslova tablice
					\hline \SetCell[c=3]{c}{\textbf{korisnik - ime tablice}}	 \\ \hline[3pt]
					\SetCell{LightGreen}IDKorisnik & INT	&  	Lorem ipsum dolor sit amet, consectetur adipiscing elit, sed do eiusmod  	\\ \hline
					korisnickoIme	& VARCHAR &   	\\ \hline 
					email & VARCHAR &   \\ \hline 
					ime & VARCHAR	&  		\\ \hline 
					\SetCell{LightBlue} primjer	& VARCHAR &   	\\ \hline 
				\end{longtblr}
				
				
			
			\subsection{Dijagram baze podataka}
				\textit{ U ovom potpoglavlju potrebno je umetnuti dijagram baze podataka. Primarni i strani ključevi moraju biti označeni, a tablice povezane. Bazu podataka je potrebno normalizirati. Podsjetite se kolegija "Baze podataka".}
			
			\eject
			
			
		\section{Dijagram razreda}
		
			\textit{Potrebno je priložiti dijagram razreda s pripadajućim opisom. Zbog preglednosti je moguće dijagram razlomiti na više njih, ali moraju biti grupirani prema sličnim razinama apstrakcije i srodnim funkcionalnostima.}\\
			
			\textbf{\textit{dio 1. revizije}}\\
			
			\textit{Prilikom prve predaje projekta, potrebno je priložiti potpuno razrađen dijagram razreda vezan uz \textbf{generičku funkcionalnost} sustava. Ostale funkcionalnosti trebaju biti idejno razrađene u dijagramu sa sljedećim komponentama: nazivi razreda, nazivi metoda i vrste pristupa metodama (npr. javni, zaštićeni), nazivi atributa razreda, veze i odnosi između razreda.}\\
			
			\textbf{\textit{dio 2. revizije}}\\			
			
			\textit{Prilikom druge predaje projekta dijagram razreda i opisi moraju odgovarati stvarnom stanju implementacije}
			
			
			
			\eject
		
		\section{Dijagram stanja}
			
			
			\textbf{\textit{dio 2. revizije}}\\
			
			\textit{Potrebno je priložiti dijagram stanja i opisati ga. Dovoljan je jedan dijagram stanja koji prikazuje \textbf{značajan dio funkcionalnosti} sustava. Na primjer, stanja korisničkog sučelja i tijek korištenja neke ključne funkcionalnosti jesu značajan dio sustava, a registracija i prijava nisu. }
			
			
			\eject 
		
		\section{Dijagram aktivnosti}
			
			\textbf{\textit{dio 2. revizije}}\\
			
			 \textit{Potrebno je priložiti dijagram aktivnosti s pripadajućim opisom. Dijagram aktivnosti treba prikazivati značajan dio sustava.}
			
			\eject
		\section{Dijagram komponenti}
		
			\textbf{\textit{dio 2. revizije}}\\
		
			 \textit{Potrebno je priložiti dijagram komponenti s pripadajućim opisom. Dijagram komponenti treba prikazivati strukturu cijele aplikacije.}